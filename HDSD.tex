\documentclass[12pt,a4paper,twoside]{article}
\usepackage[utf8]{inputenc}
\usepackage[vietnamese]{babel}
\usepackage[left=2cm,right=2cm,top=2cm,bottom=2cm]{geometry}
\usepackage{xcolor}
\usepackage[breakable,most,many,skins]{tcolorbox}
\usepackage{ntheorem}
\usepackage[loigiai]{ex_test}
\usepackage{set_box}
\usepackage{indentfirst}
\usepackage{lipsum}
\usepackage{hyperref}
\usepackage{pdfpages}
\usepackage{tikz}
\usetikzlibrary{shapes.misc,shadows}
\usepackage{newverbs}
\renewcommand{\verb}{\collectverb{\color{blue}}}
\newcommand{\addzero}[1]{\ifnum#1<10 0\fi#1}
\newcommand{\currentdate}{\the\year.\addzero{\the\month}.\addzero{\the\day}}
%------------ Gán khung cho verbatim ----------
\newtcolorbox{codeBox}{
	breakable,
	enhanced,
	boxrule=0.4mm,
	top=4mm,left=0mm,right=0mm,bottom=0mm,
	colback=yellow!15,
	colframe=brown}
\BeforeBeginEnvironment{verbatim}{\begin{codeBox}}
\AfterEndEnvironment{verbatim}{\end{codeBox}}
%--------- Tạo môi trường Định nghĩa ----------
\newtheorem{dn}{Định nghĩa}
\setTheoBox{dn}{2}{\bf Định nghĩa}
%----------- Tạo môi trường Ví dụ -------------
\newtheorem*{vd}{Ví dụ}
\setTheoBox*{vd}{4}{\bf Ví dụ}
%----------- Tạo môi trường Ví dụ -------------
\newtheorem*{nx}{Nhận xét}
\setTheoBox*{nx}{\bgBOXnew}{Nhận xét}
\hideboxInTcb{nx}
%---------- Tạo môi trường Định lý ------------
\newtheorem{dl}{Định lý}
\setTheoBox{dl}{11}{\bf Định lý}
%-------- Tạo môi trường Bài tập mẫu ----------
\newtheorem{btm}{Bài tập mẫu}
\setTheoBox{btm}{2}{\bf Bài tập mẫu}
%--------- Tạo môi trường Khởi động -----------
\newtheorem{kd}{Khởi động}
\setTheoBox{kd}{14}{\bf Khởi động}
\setboxColframe{kd}{red}
\setboxColback{kd}{yellow!20}
\setboxColbacktitle{kd}{orange}
%--------- Tạo môi trường Tính chất -----------
\newtheorem{tc}{Tính chất}
\setTheoBox{tc}{1}{\bf Tính chất}
\setboxColframe{tc}{brown}
\setboxColback{tc}{blue!10}
%-----------Khai báo môi trường lần 3----------
\newtheorem{theorem0}{<Kiểu 0>}
\setTheoBox{theorem0}{0}{\bf Kiểu số 0}
\newtheorem{theorem1}{<Kiểu 1>}
\setTheoBox{theorem1}{1}{\bf Kiểu số 1}
\newtheorem{theorem2}{<Kiểu 2>}
\setTheoBox{theorem2}{2}{\bf Kiểu số 2}
\newtheorem{theorem3}{<Kiểu 3>}
\setTheoBox{theorem3}{3}{\bf Kiểu số 3}
\newtheorem{theorem4}{<Kiểu 4>}
\setTheoBox{theorem4}{4}{\bf Kiểu số 4}
\newtheorem{theorem5}{<Kiểu 5>}
\setTheoBox{theorem5}{5}{\bf Kiểu số 5}
\newtheorem{theorem6}{<Kiểu 6>}
\setTheoBox{theorem6}{6}{\bf Kiểu số 6}
\newtheorem{theorem7}{<Kiểu 7>}
\setTheoBox{theorem7}{7}{\bf Kiểu số 7}
\newtheorem{theorem8}{<Kiểu 8>}
\setTheoBox{theorem8}{8}{\bf Kiểu số 8}
\newtheorem{theorem9}{<Kiểu 9>}
\setTheoBox{theorem9}{9}{\bf Kiểu số 9}
\newtheorem{theorem10}{<Kiểu 10>}
\setTheoBox{theorem10}{10}{\bf Kiểu số 10}
\newtheorem{theorem11}{<Kiểu 11>}
\setTheoBox{theorem11}{11}{\bf Kiểu số 11}
\newtheorem{theorem12}{<Kiểu 12>}
\setTheoBox{theorem12}{12}{\bf Kiểu số 12}
\newtheorem{theorem13}{<Kiểu 13>}
\setTheoBox{theorem13}{13}{\bf Kiểu số 13}
\newtheorem{theorem14}{<Kiểu 14>}
\setTheoBox{theorem14}{14}{\bf Kiểu số 14}
\newtheorem{theorem15}{<Kiểu 15>}
\setTheoBox{theorem15}{15}{\bf Kiểu số 15}
\newtheorem{theorem16}{<Kiểu 16>}
\setTheoBox{theorem16}{16}{\bf Kiểu số 16}
\newtheorem{theorem17}{<Kiểu 17>}
\setTheoBox{theorem17}{17}{\bf Kiểu số 17}
\newtheorem{theorem18}{<Kiểu 18>}
\setTheoBox{theorem18}{18}{\bf Kiểu số 18}
\newtheorem{theorem19}{<Kiểu 19>}
\setTheoBox{theorem19}{19}{\bf Kiểu số 19}
\newtheorem{theorem20}{<Kiểu 20>}
\setTheoBox{theorem20}{20}{\bf Kiểu số 20}
\newtheorem{theorem21}{<Kiểu 21>}
\setTheoBox{theorem21}{21}{\bf Kiểu số 21}
\newtheorem{theorem22}{<Kiểu 22>}
\setTheoBox{theorem22}{22}{\bf Kiểu số 22}
\newtheorem{theorem23}{<Kiểu 23>}
\setTheoBox{theorem23}{23}{\bf Kiểu số 23}
\newtheorem{theorem24}{<Kiểu 24>}
\setTheoBox{theorem24}{24}{\bf Kiểu số 24}
\newtheorem{theorem25}{<Kiểu 25>}
\setTheoBox{theorem25}{25}{\bf Kiểu số 25}
\newtheorem{theorem26}{<Kiểu 26>}
\setTheoBox{theorem26}{26}{\bf Kiểu số 26}
\newtheorem{theorem27}{<Kiểu 27>}
\setTheoBox{theorem27}{27}{\bf Kiểu số 27}
\newtheorem{theorem28}{<Kiểu 28>}
\setTheoBox{theorem28}{28}{\bf Kiểu số 28}
\newtheorem{theorem29}{<Kiểu 29>}
\setTheoBox{theorem29}{29}{\bf Kiểu số 29}
\newtheorem{theorem30}{<Kiểu 30>}
\setTheoBox{theorem30}{30}{\bf Kiểu số 30}
\newtheorem{theorem31}{<Kiểu 31>}
\setTheoBox{theorem31}{31}{\bf Kiểu số 31}
\newtheorem{theorem32}{<Kiểu 32>}
\setTheoBox{theorem32}{32}{\bf Kiểu số 32}
\newtheorem{theorem33}{<Kiểu 33>}
\setTheoBox{theorem33}{33}{\bf Kiểu số 33}
\newtheorem{theorem34}{<Kiểu 34>}
\setTheoBox{theorem34}{34}{\bf Kiểu số 34}
%-------------------------------------------
\newsavebox{\myverbcontent}
%---------------- Bắt đầu ------------------
\begin{document}
	\pagestyle{plain}
	\fontsize{12pt}{16pt}\selectfont
	\begin{center}
		\Large\bf\color{red!70!black}\fontfamily{lmss}\selectfont Hướng dẫn sử dụng gói lệnh set\_box.sty 1.\currentdate\par
	\end{center}
	\tableofcontents
	\newpage
	\section{Giới thiệu sơ lược về gói lệnh}
	\subsection{Nguyên nhân ra đời}
	\begin{itemize}
		\item Đơn giản hoá bước tạo các khung nội dung mới theo mẫu có sẵn.
		\item Các khung được tạo tương thích tốt với gói lệnh \texttt{ex\_test.sty} đã rất phổ biến hiện nay.
		\item Giải quyết được các vấn đề về lồng môi trường vào nhau, cũng như ẩn hiện môi trường được đóng khung.
	\end{itemize}
	\subsection{Một số lưu ý}
	\begin{itemize}
		\item Gói lệnh nên đi kèm và khai báo phía sau hai gói \texttt{ex\_test.sty} và \texttt{ntheorem.sty}. 
		\item Chỉ nên cài đặt khung cho \texttt{theorem} mới hoặc đã áp dụng khung trước đó. Hạn chế áp dụng thêm các tác động khác ngoài gói này, có thể gây lỗi.
	\end{itemize}
	\subsection{Đường dẫn cập nhật}
	\begin{itemize}
		\item Github: \url{https://github.com/vananh2801/set_box_latex/releases}
	\end{itemize}
	\newpage
	\section{Hướng dẫn sử dụng}
	\subsection{Tạo khung cho \texttt{theorem} (định nghĩa bởi gói \texttt{ntheorem})}
	\subsubsection{Giới thiệu lệnh}
	Các bước thực hiện để tạo khung như sau:
	\begin{itemize}
		\item \textbf{Bước 1.} Khai báo \texttt{theorem} bằng gói \texttt{ntheorem}.\\
		Chẳng hạn:
		\begin{verbatim}
			\newtheorem{dn}{\bf Định nghĩa}
			\newtheorem*{vd}{\bf Ví dụ}
		\end{verbatim}
		\item \textbf{Bước 2.} Dùng lệnh:
		\begin{itemize}
			\item
			\begin{lrbox}{\myverbcontent}
				\verb|\setTheoBox{<tên theorem>}{<kiểu>}{<Tiêu đề>}|
			\end{lrbox}
			\fbox{\usebox{\myverbcontent}}: tạo khung đánh số.
			\item 
			\begin{lrbox}{\myverbcontent}
				\verb|\setTheoBox*{<tên theorem>}{<kiểu>}{<Tiêu đề>}|
			\end{lrbox}
			\fbox{\usebox{\myverbcontent}}: tạo khung không đánh số.
		\end{itemize}
		Chẳng hạn:
		\begin{verbatim}
			\setTheoBox{dn}{2}{\bf Định nghĩa} % có đánh số thứ tự
			\setTheoBox*{vd}{4}{\bf Ví dụ} % không đánh số thứ tự 
		\end{verbatim}
		\item \textbf{Bước 3.} Sử dụng theo cấu trúc như sau:
		\begin{verbatim}
			\begin{dn}[Tên định nghĩa]
				Nội dung Định nghĩa...
			\end{dn}
			\begin{vd}
				Đề bài...
				\loigiai{
					Lời giải...
				}
			\end{vd}
		\end{verbatim}
	\end{itemize}\par
	Hiện tại gói lệnh \texttt{set\_box} có sẵn $35$ kiểu (từ $0$ đến $34$), được liệt kê ở mục $3$. Trong đó, kiểu số 0 là kiểu hiển thị không có khung. Đối với các \texttt{theorem} mà thầy cô không muốn tạo khung thì hãy dùng kiểu số $0$ \textit{(đây là bước bắt buộc)}.
	\newpage
	\subsubsection{Minh hoạ}
	\begin{verbatim}
		\documentclass[12pt,a4paper,twoside]{article}
		\usepackage[utf8]{vietnam}
		\usepackage[left=2cm,right=2cm,top=2cm,bottom=2cm]{geometry}
		\usepackage{ntheorem}
		\usepackage[loigiai]{ex_test}
		\usepackage{set_box}
		%---------- Định nghĩa -----------
		\newtheorem{dn}{Định nghĩa}
		\setTheoBox{dn}{2}{\bf Định nghĩa}
		%------------- Ví dụ -------------
		\newtheorem{vd}*{Ví dụ}
		\setTheoBox{vd}*{4}{\bf Ví dụ}
		%-------- Nội dung chính ---------
		\begin{document}
			\begin{dn}[Tên định nghĩa]
				Nội dung Định nghĩa...
			\end{dn}
			\begin{vd}
				Đề bài...
				\loigiai{
					Lời giải...
				}
			\end{vd}
		\end{document}
	\end{verbatim}
	\subsubsection{Kết quả thu được}
	\begin{dn}[Tên định nghĩa]
		Nội dung Định nghĩa...
	\end{dn}
	\begin{vd}
		Đề bài...
		\loigiai{
			Lời giải...
		}
	\end{vd}
	\newpage
	\subsubsection{Khai báo kiểu khung cá nhân}
	Gói lệnh \texttt{set\_box} có hỗ trợ khung riêng do thầy cô tự khai báo. Ta cần chú ý các lệnh sau:
	\begin{center}
		\begin{tabular}{|m{5cm}|m{4.5cm}|m{5.8cm}|}
			\hline
			\hfil\textbf{Lệnh}&\hfil\textbf{Giải thích}&\hfil\textbf{Minh hoạ}
			\\\hline
			\verb|\sb@labelthm|&tiêu đề&\textbf{Định nghĩa}
			\\\hline
			\verb|\sb@Currentlabel|&số thứ tự&\textbf{1}
			\\\hline
			\verb|\sb@sublabelthm|&tiêu đề phụ&\textbf{Tên định nghĩa}
			\\\hline
			\verb|\sb@sublabelthmBracket|&tiêu đề phụ trong ngoặc&\textbf{(Tên định nghĩa)}
			\\\hline
			\verb|\sb@labelthmshort|&tiêu đề + số thứ tự&\textbf{Định nghĩa 1}
			\\\hline
			\verb|\sb@labelthmfull|&tiêu đề + số thứ tự + tiêu đề phụ&\textbf{Định nghĩa 1 (Tên tiêu đề)}
			\\\hline
			\verb|\sb@labelthmshortDot|&tiêu đề + số thứ tự + dấu chấm&\textbf{Định nghĩa 1.}
			\\\hline
			\verb|\sb@labelthmfullDot|&tiêu đề + số thứ tự + tiêu đề phụ + dấu chấm&\textbf{Định nghĩa 1 (Tên tiêu đề).}
			\\\hline
		\end{tabular}
	\end{center}\par
	Thầy cô khai báo kiểu khung riêng nằm ngoài \texttt{set\_box}. Vì trong khai báo có chứa dấu a còng \verb|@| nên ta phải khai báo trong cặp lệnh \verb|\makeatletter| và \verb|\makeatother|. Sau đây là cấu trúc chuẩn dùng cho gói \texttt{set\_box}:
	\begin{verbatim}
		\makeatletter
		\newcommand{<tên kiểu>}{%
				%%% Khai báo khung.
				\def\sb@title{<Định dạng 1>}
				\def\sb@beginbox{%
						\begin{tcolorbox}[
								<Định dạng 2>
								]
						}
				\def\endbox{%<Định dạng 3>\end{tcolorbox}}
				%%% Nếu trong tcolorbox khác thì sẽ mất khung.
				\ifsb@InTcolorbox
						\Ifsb@hideboxInTcb
								\def\sb@title{<Định dạng 4>}
								\def\sb@beginbox{<Định dạng 5>}
								\def\endbox{<Định dạng 6>}
						\fi
				\fi
		}
	\makeatother
	\end{verbatim}
	Cấu trúc chuẩn gồm 2 phần chính: khai báo ``định dạng khung'' và khai báo ``định dạng thay thế'' nếu dùng chức năng ẩn khung khi trong tcolorbox khác (xem thêm ở mục 2.3).\par 
	\newpage
	Ở mỗi phần này, chúng ta để phải khai báo đầy đủ 3 \texttt{macro} như sau:
	\begin{itemize}
		\item \verb|\sb@title|: Tiêu đề xuất hiện ở đầu đoạn nội dung.
		\begin{itemize}
			\item Khi ta cần hiện tiêu đề nằm trên khung, thì ta phải ẩn tiêu đề ở đầu đoạn nội dung. Ta khai báo \verb|\def\sb@title{}|.
			\item Ngược lại, ta cần hiện tiêu đề ở đầu đoạn nội dung. Ta khai báo\\
			\verb|\def\sb@title{\sb@labelthmfullDot}|\\
			hoặc \verb|\def\sb@title{\sb@labelthmshortDot}|
			\item Khai báo \verb|\sb@title| đầy đủ để giúp lệnh \verb|\immini| và \verb|\sochc| có thể hoạt động đúng.
		\end{itemize}
		\item \verb|\sb@beginbox|: 
		\begin{itemize}
			\item Ta khai báo định dạng \texttt{tcolorbox} trong cặp $[...]$ của lệnh \verb|\begin{tcolorbox}|.
			\item Nếu muốn mất khung như kiểu số $0$ thì ta khai báo \verb|\def\sb@beginbox{}|
		\end{itemize}
		\item \verb|\endbox|:
		\begin{itemize}
			\item Nếu ta đã có dùng \verb|\begin{tcolorbox}| thì \textbf{bắt buộc} phải có \verb|\end{tcolorbox}| ở cuối nội dung khai báo \verb|\def\endbox{...}|. Nếu ta không cần có tiêu đề phụ ở cuối khung thì ta chỉ cần khai báo \verb|\def\endbox{\end{tcolorbox}}| là được. Ngược lại ta cần hiện tiêu đề phụ ở cuối (thường dùng để ghi nguồn ví dụ, bài tập,...) thì ta có thể khai báo\\
			\verb|\def\endbox{\par\noindent\hfill|\\
			\verb|\textit{\sb@sublabelthmBracket}\end{tcolorbox}}|
			\item Nếu không khai báo \verb|\begin{tcolorbox}| thì ta không cần \verb|\end{tcolorbox}|. Làm tương tự như trên nhưng không chèn \verb|\end{tcolorbox}|. Chẳng hạn, \verb|\def\endbox{}| hoặc \\
			\verb|\def\endbox{\par\noindent\hfill|\verb|\textit{\sb@sublabelthmBracket}}|
		\end{itemize}
	\end{itemize}\par
	Gói \texttt{set\_box} đã khai báo sẵn lệnh \verb|\bgBOXnew| dựa trên \verb|\bgBOX| của gói \texttt{ex\_test}. Ngoài ra thầy cô có thể tham khảo cách khai báo bên trong file \texttt{set\_box.sty}.
	\begin{verbatim}
		\makeatletter
		\newcommand{\bgBOXnew}{%
				\def\sb@title{}
				\def\sb@beginbox{
						\begin{tcolorbox}[%
								enhanced,
								breakable,
								drop fuzzy shadow southeast,
								before skip=4mm,
								after skip=4mm,
								colback=yellow!7,
								colframe=red!50!black,
								boxrule=1pt,
								attach boxed title to top left={
										xshift=1cm,yshift*=1mm-\tcboxedtitleheight},
										boxed title style={frame code={
												\path[fill=red!30!black]
												([yshift=-1mm,xshift=-1mm]frame.north west)
												arc[start angle=0,end angle=180,radius=1mm]
												([yshift=-1mm,xshift=1mm]frame.north east)
												arc[start angle=180,end angle=0,radius=1mm];
												\path[left color=red!60!black, 
												right color=red!60!black,
												middle color=red!85!black]
												([xshift=-2mm]frame.north west) -- 
												([xshift=2mm]frame.north east)
												[rounded corners=1mm]-- 
												([xshift=1mm,yshift=-1mm]frame.north east)
												-- (frame.south east) -- (frame.south west)
												-- ([xshift=-1mm,yshift=-1mm]frame.north west)
												[sharp corners]-- cycle;
												},
										interior engine=empty},
								fonttitle=\color{white}\bf\fontfamily{qag}\selectfont,
								title={\sb@labelthmshort}
								]
						}
				\def\endbox{\par\noindent\hfill\bf\color{red!30!black}
								\fontfamily{qag}\selectfont\sb@sublabelthmBracket\end{tcolorbox}}
				%%% Nếu trong tcolorbox khác thì sẽ mất khung.
				\ifsb@InTcolorbox
						\Ifsb@hideboxInTcb
								\def\sb@title{\bf\color{red!30!black}
								\fontfamily{qag}\selectfont\sb@labelthmshortDot}
								\def\sb@beginbox{}
								\def\endbox{\par\noindent\hfill\bf\color{red!30!black}
								\fontfamily{qag}\selectfont\sb@sublabelthmBracket}
						\fi
				\fi
		}
	\end{verbatim}
	Ở tham số thứ 2 của lệnh \verb|\setTheoBox|, thầy cô không dùng số mà dùng câu lệnh vừa khai báo:
	\begin{verbatim}
		% Đã định dạng chữ trong lệnh \bgBOXnew
		\setTheoBox*{nx}{\bgBOXnew}{Nhận xét}
	\end{verbatim}
	\newpage
	Ta thu được kết quả như sau khi Nhận xét nằm \textbf{bên ngoài} với Định nghĩa:
	\begin{dn}
		Nội dung định nghĩa...
	\end{dn}
	\begin{nx}[Nguồn]
		Nội dung nhận xét...
	\end{nx}
	Ta thu được kết quả như sau khi Nhận xét \textbf{bên trong} Định nghĩa (xem thêm ở mục 2.3):
	\begin{dn}
		Nội dung định nghĩa...
		\begin{nx}[Nguồn]
			Nội dung nhận xét...
		\end{nx}
	\end{dn}
	\subsection{Thay đổi màu khung và màu nền}
	\subsubsection{Giới thiệu lệnh}
	Để đổi màu mặc định, trước \verb|\setTheoBox|, ta dùng các lệnh sau:
	\begin{itemize}
		\begin{lrbox}{\myverbcontent}
			\verb|\setboxColframeSetDefault{<màu>}|
		\end{lrbox}
		\item Lệnh \fbox{\usebox{\myverbcontent}}: đổi màu khung mặc định.
		\begin{lrbox}{\myverbcontent}
			\verb|\setboxColbackSetDefault{<màu>}|
		\end{lrbox}
		\item Lệnh \fbox{\usebox{\myverbcontent}}: đổi màu nền mặc định.
		\begin{lrbox}{\myverbcontent}
			\verb|\setboxColbacktitleSetDefault{<màu>}|
		\end{lrbox}
		\item Lệnh \fbox{\usebox{\myverbcontent}}: đổi màu nền tiêu đề mặc định.
	\end{itemize}\par
	Để đổi màu riêng, sau \verb|\setTheoBox|, ta dùng các lệnh sau:
	\begin{itemize}
		\begin{lrbox}{\myverbcontent}
			\verb|\setboxColframe{<tên môi trường>}{<màu>}|
		\end{lrbox}
		\item Lệnh \fbox{\usebox{\myverbcontent}}: đổi màu khung của môi trường.
		\begin{lrbox}{\myverbcontent}
			\verb|\setboxColback{<tên môi trường>}{<màu>}|
		\end{lrbox}
		\item Lệnh \fbox{\usebox{\myverbcontent}}: đổi màu nền của môi trường.
		\begin{lrbox}{\myverbcontent}
			\verb|\setboxColbacktitle{<tên môi trường>}{<màu>}|
		\end{lrbox}
		\item Lệnh \fbox{\usebox{\myverbcontent}}: đổi màu nền tiêu đề của môi trường.
	\end{itemize}\par
	Cần lưu ý rằng các lệnh này chỉ áp dụng với các kiểu khung có sẵn của \texttt{set\_box}. Nếu thầy cô muốn áp dụng trên kiểu của thầy cô tự tạo thì nên dùng các lệnh sau để khai báo màu:
	\begin{center}
		\begin{tabular}{|m{4cm}|m{4.5cm}|m{6.8cm}|}
			\hline
			\hfil\textbf{Lệnh}&\hfil\textbf{Giải thích}&\hfil\textbf{Minh hoạ}
			\\\hline
			\verb|\sb@colbacktitle|&màu nền tiêu đề&\verb|colbacktitle=\sb@colbacktitle|
			\\\hline
			\verb|\sb@colback|&màu nền của khung&\verb|colback=\sb@colback|
			\\\hline
			\verb|\sb@colframe|&màu khung&\verb|colframe=\sb@colframe|
			\\\hline
		\end{tabular}
	\end{center}\par
	\newpage
	\subsubsection{Minh hoạ}
	\begin{verbatim}
		\documentclass[12pt,a4paper,twoside]{article}
		\usepackage[utf8]{vietnam}
		\usepackage[left=2cm,right=2cm,top=2cm,bottom=2cm]{geometry}
		\usepackage{ntheorem}
		\usepackage[loigiai]{ex_test}
		\usepackage{set_box}
		%--------- Tạo môi trường Khởi động -----------
		\newtheorem{kd}{Khởi động}
		\setTheoBox{kd}{14}{\bf Khởi động}
		\setboxColframe{kd}{red}
		\setboxColback{kd}{yellow!20}
		\setboxColbacktitle{kd}{orange}
		%--------- Tạo môi trường Tính chất -----------
		\newtheorem{tc}{Tính chất}
		\setTheoBox{tc}{1}{\bf Tính chất}
		\setboxColframe{tc}{brown}
		\setboxColback{tc}{blue!10}
		%------- Nội dung chính --------
		\begin{document}
			\begin{kd}
				Nội dung Khởi động...
			\end{kd}
			\begin{tc}[Tên tính chất]
				Nội dung Tính chất...
			\end{tc}
		\end{document}
	\end{verbatim}
	\subsubsection{Kết quả thu được}
	\begin{kd}
		Nội dung Khởi động...
	\end{kd}
	\begin{tc}[Tên tính chất]
		Nội dung Tính chất...
	\end{tc}
	\newpage
	\subsection{Ẩn khung tự động khi sử dụng môi trường bên trong \texttt{tcolorbox}}
	\subsubsection{Giới thiệu lệnh}
	\begin{itemize}
		\begin{lrbox}{\myverbcontent}
			\verb|\showboxInTcb{<tên môi trường>}|
		\end{lrbox}
		\item Lệnh \fbox{\usebox{\myverbcontent}} (mặc định): hiện ``khung'' khi sử dụng môi trường trong \verb|\begin{tcolorbox}...\end{tcolorbox}|.
		\begin{lrbox}{\myverbcontent}
			\verb|\hideboxInTcb{<tên môi trường>}|
		\end{lrbox}
		\item Lệnh \fbox{\usebox{\myverbcontent}}: ẩn ``khung'' khi sử dụng môi trường trong trong \verb|\begin{tcolorbox}...\end{tcolorbox}|.
	\end{itemize}\par
	\subsubsection{Minh hoạ}
	\begin{verbatim}
		\hideboxInTcb{dl} % Ẩn khung
		\hideboxInTcb{btm} % Ẩn khung
		\begin{tcolorbox}[frame empty,colback=red!5]
			\begin{dl}[Tên định lý]
				Nội dung Định lý...
			\end{dl}
			\begin{btm}
				Đề bài...
				\loigiai{
					Lời giải...
				}
			\end{btm}
		\end{tcolorbox}
		\showboxInTcb{dl} % Hiện khung
		\showboxInTcb{btm} % Hiện khung
		\begin{tcolorbox}[frame empty,colback=red!5]
			\begin{dl}[Tên định lý]
				Nội dung Định lý...
			\end{dl}
			\begin{btm}
				Đề bài...
				\loigiai{
					Lời giải...
				}
			\end{btm}
		\end{tcolorbox}
	\end{verbatim}
	\newpage
	\subsubsection{Kết quả thu được}
	\setcounter{dl}{0}
	\setcounter{btm}{0}
	\hideboxInTcb{dl}
	\hideboxInTcb{btm}
	\begin{tcolorbox}[frame empty,colback=red!5]
		\begin{dl}[Tên định lý]
			Nội dung Định lý...
		\end{dl}
		\begin{btm}
			Đề bài...
			\loigiai{
				Lời giải...
			}
		\end{btm}
	\end{tcolorbox}
	\showboxInTcb{dl}
	\showboxInTcb{btm}
	\begin{tcolorbox}[frame empty,colback=red!5]
		\begin{dl}[Tên định lý]
			Nội dung Định lý...
		\end{dl}
		\begin{btm}
			Đề bài...
			\loigiai{
				Lời giải...
			}
		\end{btm}
	\end{tcolorbox}
	\subsubsection{Lưu ý}
	Để tạo một \texttt{tcolorbox} nền mà các lệnh \verb|\showboxInTcb| và \verb|\hideboxInTcb| \textit{có thể} tác động lên các môi trường bên trong, ta có thể thử sử dụng cấu trúc khai báo sau để chèn \verb|\begin{tcolorbox}| và \verb|\end{tcolorbox}|:
	\begin{verbatim}
		% Khai báo bắt đầu tcolorbox cho môi trường nen (Nền)
		\def\beginNen{
			\begin{tcolorbox}[<option>]
			}
		% Khai báo kết thúc tcolorbox cho môi trường nen (Nền)
		\def\endNen{ % Không nên dùng \endbox
			\end{tcolorbox}
		}
		% Tạo môi trường nen (Nền)
		\newenvironment{nen}{\beginNen}{\endNen}
	\end{verbatim}
	Để tạo một \texttt{tcolorbox} nền mà các lệnh \verb|\showboxInTcb| và \verb|\hideboxInTcb| \textit{không thể} tác động lên các môi trường bên trong, ta có thể thử sử dụng cấu trúc khai báo sau để chèn \verb|\begin{tcolorbox}| và \verb|\end{tcolorbox}|:
	\begin{verbatim}
		% Tạo tcolorbox mới nen (Nền)
		\newtcolorbox{nen}{<option>}
	\end{verbatim}
	\subsection{Ẩn môi trường}
	\subsubsection{Giới thiệu lệnh}
	\begin{itemize}
		\begin{lrbox}{\myverbcontent}
			\verb|\setboxHide{<tên môi trường>}|
		\end{lrbox}
		\item Lệnh \fbox{\usebox{\myverbcontent}}: ẩn môi trường hoàn toàn kể từ khi áp dụng lệnh này.
		\begin{lrbox}{\myverbcontent}
			\verb|\setboxHide[<danh sách>]{<tên môi trường>}|
		\end{lrbox}
		\item Lệnh \fbox{\usebox{\myverbcontent}}: ẩn môi trường theo danh sách kể tử khi áp dụng lệnh này.
	\end{itemize}
	\subsubsection{Lưu ý}
	\begin{itemize}
		\item Chỉ số sau khi ẩn vẫn đánh giống như chỉ số gốc khi không ẩn.
		\item Hai lệnh này là lệnh mới của gói \texttt{set\_box}, không phải là lệnh \verb|\hideenviron| và \verb|\print| của \texttt{ex\_test}, chỉ áp dụng với các môi trường đã được xử lý bởi \verb|\setTheoBox| hoặc \verb|\setEnvBox|.
		\item Khi cần ẩn các câu hỏi, ta có thể khai báo danh sách tương tự khi dùng \verb|\foreach|.Chẳng hạn, danh sách gồm các số từ 3 đến 5 và 10 đến 15 thì ta dùng
		\begin{center}
			\verb|\setboxHide[3,...,5,10,...,15]{<tên môi trường>}|.
		\end{center}
	\end{itemize}\par
	\subsubsection{Minh hoạ}
	\begin{verbatim}
		\setcounter{dn}{0}
		\setboxHide{dn} % Lệnh ẩn môi trường Định nghĩa
		\begin{dn}
			Nội dung của Định nghĩa đầu tiên...
		\end{dn}
		\begin{dn}
			Nội dung của Định nghĩa thứ hai...
		\end{dn}
		\setcounter{dn}{0}
		\setboxHide[2,4]{dn} % Lệnh ẩn môi trường Định nghĩa
		\begin{dn}
			Nội dung của Định nghĩa 1...
		\end{dn}
		\begin{dn}
			Nội dung của Định nghĩa 2...
		\end{dn}
		\begin{dn}
			Nội dung của Định nghĩa 3...
		\end{dn}
		\begin{dn}
			Nội dung của Định nghĩa 4...
		\end{dn}
	\end{verbatim}
	\subsubsection{Kết quả thu được}
	\setcounter{dn}{0}
	\setboxHide{dn} % Lệnh ẩn Định nghĩa
	\begin{dn}
		Nội dung của Định nghĩa đầu tiên...
	\end{dn}
	\setcounter{dn}{0}
	\setboxHide[2,4]{dn} % Lệnh ẩn Định nghĩa
	\begin{dn}
		Nội dung của Định nghĩa 1...
	\end{dn}
	\begin{dn}
		Nội dung của Định nghĩa 2...
	\end{dn}
	\begin{dn}
		Nội dung của Định nghĩa 3...
	\end{dn}
	\begin{dn}
		Nội dung của Định nghĩa 4...
	\end{dn}
	\newpage
	\section{Các kiểu khung có sẵn ở phiên bản hiện tại}
	\begin{theorem0}
		Nội dung của kiểu 0...
	\end{theorem0}
	\vspace{0.6cm}
	\begin{theorem1}
		Nội dung của kiểu 1...
	\end{theorem1}
	\vspace{0.6cm}
	\begin{theorem2}
		Nội dung của kiểu 2...
	\end{theorem2}
	\vspace{0.6cm}
	\begin{theorem3}
		Nội dung của kiểu 3...
	\end{theorem3}
	\vspace{0.6cm}
	\begin{theorem4}
		Nội dung của kiểu 4...
	\end{theorem4}
	\vspace{0.6cm}
	\begin{theorem5}
		Nội dung của kiểu 5...
	\end{theorem5}
	\vspace{0.6cm}
	\begin{theorem6}
		Nội dung của kiểu 6...
	\end{theorem6}
	\vspace{0.6cm}
	\begin{theorem7}
		Nội dung của kiểu 7...
	\end{theorem7}
	\vspace{0.6cm}
	\begin{theorem8}
		Nội dung của kiểu 8...
	\end{theorem8}
	\vspace{0.6cm}
	\begin{theorem9}
		Nội dung của kiểu 9...
	\end{theorem9}
	\vspace{0.6cm}
	\begin{theorem10}
		Nội dung của kiểu 10...
	\end{theorem10}
	\vspace{0.6cm}
	\begin{theorem11}
		Nội dung của kiểu 11...
	\end{theorem11}
	\vspace{0.6cm}
	\begin{theorem12}
		Nội dung của kiểu 12...
	\end{theorem12}
	\vspace{0.6cm}
	\begin{theorem13}
		Nội dung của kiểu 13...
	\end{theorem13}
	\vspace{0.6cm}
	\begin{theorem14}
		Nội dung của kiểu 14...
	\end{theorem14}
	\vspace{0.6cm}
	\begin{theorem15}
		Nội dung của kiểu 15...
	\end{theorem15}
	\vspace{0.6cm}
	\begin{theorem16}
		Nội dung của kiểu 16...
	\end{theorem16}
	\vspace{0.6cm}
	\begin{theorem17}
		Nội dung của kiểu 17...
	\end{theorem17}
	\vspace{0.6cm}
	\begin{theorem18}
		Nội dung của kiểu 18...
	\end{theorem18}
	\vspace{0.6cm}
	\begin{theorem19}
		Nội dung của kiểu 19...
	\end{theorem19}
	\vspace{0.6cm}
	\begin{theorem20}
		Nội dung của kiểu 20...
	\end{theorem20}
	\vspace{0.6cm}
	\begin{theorem21}
		Nội dung của kiểu 21...
	\end{theorem21}
	\vspace{0.6cm}
	\begin{theorem22}
		Nội dung của kiểu 22...
	\end{theorem22}
	\vspace{0.6cm}
	\begin{theorem23}
		Nội dung của kiểu 23...
	\end{theorem23}
	\vspace{0.6cm}
	\begin{theorem24}
		Nội dung của kiểu 24...
	\end{theorem24}
	\vspace{0.6cm}
	\begin{theorem25}
		Nội dung của kiểu 25...
	\end{theorem25}
	\vspace{0.6cm}
	\begin{theorem26}
		Nội dung của kiểu 26...
	\end{theorem26}
	\vspace{0.6cm}
	\begin{theorem27}
		Nội dung của kiểu 27...
	\end{theorem27}
	\vspace{0.6cm}
	\begin{theorem28}
		Nội dung của kiểu 28...
	\end{theorem28}
	\vspace{0.6cm}
	\begin{theorem29}
		Nội dung của kiểu 29...
	\end{theorem29}
	\begin{theorem30}
		Nội dung của kiểu 30...
	\end{theorem30}
	\begin{theorem31}
		Nội dung của kiểu 31...
	\end{theorem31}
	\begin{theorem32}
		Nội dung của kiểu 32...
	\end{theorem32}
	\begin{theorem33}
		Nội dung của kiểu 33...
	\end{theorem33}
	\begin{theorem34}
		Nội dung của kiểu 34...
	\end{theorem34}
\end{document}