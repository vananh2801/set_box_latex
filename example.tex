\documentclass[12pt,a4paper,twoside]{article}
\usepackage[utf8]{vietnam}
\usepackage[left=2cm,right=2cm,top=2cm,bottom=2cm]{geometry}
\usepackage{xcolor}
\usepackage{ntheorem}
\usepackage[loigiai]{ex_test}
\usepackage{set_box}
\usepackage{indentfirst}
\usepackage{lipsum}
\usepackage{needspace}
%---- Cài đặt màu mặc định gói set_box ----
\setboxColframeSetDefault{brown}
\setboxColbackSetDefault{black!5}
\setboxColbacktitleSetDefault{brown!30}
%----------- Định dạng theorem ------------
\theoremstyle{immini}
\theoremseparator{.}
\theoremheaderfont{\color{brown!70!black}\bf\fontfamily{lmss}\selectfont}
%--------------- Định nghĩa ---------------
\newtheorem{dn}{Định nghĩa}
\setTheoBox{dn}{24}{\color{brown!70!black}\bf\fontfamily{lmss}\selectfont Định nghĩa}
%--------------- Tính chất ---------------
\newtheorem{tc}{Tính chất}
\setTheoBox{tc}{27}{\color{brown!70!black}\bf\fontfamily{lmss}\selectfont Tính chất}
%----------------- Ví dụ -----------------
\newtheorem{vd}{Ví dụ}
\setTheoBox{vd}{23}{\color{brown!70!black}\bf\fontfamily{lmss}\selectfont Ví dụ}
%----------------- Dạng- -----------------
\newtheorem{dang}{Dạng}
\setTheoBox{dang}{12}{\color{brown!70!black}\bf\fontfamily{lmss}\selectfont Dạng}
%---------------- Ghi nhớ ----------------
\newtheorem{ghinho}{Ghi nhớ}
\setTheoBox{ghinho}{22}{\color{brown!70!black}\bf\fontfamily{lmss}\selectfont Ghi nhớ}
%------------------- ex ------------------
\renewtheorem{ex}{Câu}
\setTheoBox{ex}{14}{\color{brown!70!black}\bf\fontfamily{lmss}\selectfont Câu}
% Gói lệnh sẽ áp dụng tương tự đối với chc
%----------------- Lời giải --------------
\def\loigiaiEX{\needspace{1cm}\parbox[c]{\linewidth-1mm}{\centering\color{brown!70!black}\bf\fontfamily{lmss}\selectfont Lời giải.}}
%----------- Định dạng theorem ------------
\theoremseparator{:}
\theoremheaderfont{\color{brown!70!black}\bf\fontfamily{lmss}\selectfont}
\theorembodyfont{\itshape}
%----------------- Chú ý -----------------
\newtheorem*{chuy}{Chú ý}
\setTheoBox{chuy}{25}{\color{brown!70!black}\bf\fontfamily{lmss}\selectfont Chú ý}
%--------------- Nhận xét ----------------
\newtheorem*{nx}{Nhận xét}
\setTheoBox{nx}{5}{\color{brown!70!black}\bf\fontfamily{lmss}\selectfont Nhận xét}
\begin{document}
	\begin{dn}[Tên định nghĩa]
		\lipsum[1]
	\end{dn}
	\begin{dn}[Tên định nghĩa]
		\lipsum[1]
	\end{dn}
	\begin{tc}[Tên tính chất]
		\lipsum[1]
	\end{tc}
	\begin{vd}
		\lipsum[1]
		\loigiai{
			\dotlineEX{3}
		}
	\end{vd}
	\begin{vd}
		\lipsum[1]
		\loigiai{
			\dotlineEX{3}
		}
	\end{vd}
	\begin{chuy}
		\begin{itemize}
			\item Điều thứ nhất;
			\item Điều thứ hai.
		\end{itemize}
	\end{chuy}
	\begin{chuy}
		\begin{itemize}
			\item Điều thứ nhất;
			\item Điều thứ hai.
		\end{itemize}
	\end{chuy}
	\begin{nx}
		\lipsum[1]
	\end{nx}
	\begin{dang}[Tên dạng]
		\lipsum[1]
	\end{dang}
	
	\begin{ex}
		\immini[thm]{
			Nội dung câu hỏi con thứ nhất.
		}{
			\begin{tikzpicture}
				\draw (0,0) -- (3,3);
			\end{tikzpicture}
		}
		\begin{chc}
			Nội dung câu hỏi con thứ nhất.
			\loigiai{
				<Lời giải câu hỏi con thứ nhất>
			}
		\end{chc}
		\begin{chc}
			Nội dung câu hỏi con thứ hai.
			\loigiai{
				<Lời giải câu hỏi con thứ hai>
			}
		\end{chc}
	\end{ex}
	\begin{ex}
		\sochc{2}
		\immini[thm]{
			Nội dung câu hỏi con thứ nhất.
		}{
			\begin{tikzpicture}
				\draw (0,0) -- (3,3);
			\end{tikzpicture}
		}
		\begin{chc}
			Nội dung câu hỏi con thứ nhất.
			\loigiai{
				<Lời giải câu hỏi con thứ nhất>
			}
		\end{chc}
		\begin{chc}
			Nội dung câu hỏi con thứ hai.
			\loigiai{
				<Lời giải câu hỏi con thứ hai>
			}
		\end{chc}
	\end{ex}
	\begin{ghinho}
		\lipsum[1]
	\end{ghinho}
	\begin{ex}
		<Câu dẫn cho câu hỏi nhiều lựa chọn>.
		\choice
		{\True Phương án A}
		{Phương án B}
		{Phương án C}
		{Phương án D}
		\loigiai{
			...
		}
	\end{ex}
	\begin{ex}
		<Câu dẫn cho câu hỏi nhiều lựa chọn>.
		\choiceTF[1t]
		{\True Phương án A}
		{Phương án B}
		{Phương án C}
		{\True Phương án D}
		\loigiai{
			...
		}
	\end{ex}
	
\end{document}