\documentclass[12pt,a4paper,twoside]{article}
\usepackage[utf8]{inputenc}
\usepackage[vietnamese]{babel}
\usepackage[left=2cm,right=2cm,top=2cm,bottom=2cm]{geometry}
\usepackage{xcolor}
\usepackage[breakable,most,many,skins]{tcolorbox}
\usepackage{ntheorem}
\usepackage[loigiai]{ex_test}
\usepackage{set_box}
\usepackage{indentfirst}
\usepackage{lipsum}

\usepackage{pdfpages}
\usepackage{tikz}
\usetikzlibrary{shapes.misc,shadows}
%------------ Gán khung cho verbatim ----------
\newtcolorbox{codeBox}{
	breakable,
	enhanced,
	boxrule=0.4mm,
	top=4mm,left=0mm,right=0mm,bottom=0mm,
	colback=yellow!15,
	colframe=brown}
\BeforeBeginEnvironment{verbatim}{\begin{codeBox}}
\AfterEndEnvironment{verbatim}{\end{codeBox}}
%--------- Tạo môi trường Định nghĩa ----------
\newtheorem{dn}{Định nghĩa}
\setTheoBox{dn}{2}{\bf Định nghĩa}
%----------- Tạo môi trường Ví dụ -------------
\newtheorem{vd}{Ví dụ}
\setTheoBox{vd}{4}{\bf Ví dụ}
%---------- Tạo môi trường Định lý ------------
\newtheorem{dl}{Định lý}
\setTheoBox{dl}{11}{\bf Định lý}
%-------- Tạo môi trường Bài tập mẫu ----------
\newtheorem{btm}{Bài tập mẫu}
\setTheoBox{btm}{2}{\bf Bài tập mẫu}
%--------- Tạo môi trường Khởi động -----------
\newtheorem{kd}{Khởi động}
\setTheoBox{kd}{14}{\bf Khởi động}
\setboxColframe{kd}{red}
\setboxColback{kd}{yellow!20}
\setboxColbacktitle{kd}{orange}
%--------- Tạo môi trường Tính chất -----------
\newtheorem{tc}{Tính chất}
\setTheoBox{tc}{1}{\bf Tính chất}
\setboxColframe{tc}{brown}
\setboxColback{tc}{blue!10}
%-----------Khai báo môi trường lần 3----------
\newtheorem{theorem0}{<Kiểu 0>}
\setTheoBox{theorem0}{0}{\bf Kiểu số 0}
\newtheorem{theorem1}{<Kiểu 1>}
\setTheoBox{theorem1}{1}{\bf Kiểu số 1}
\newtheorem{theorem2}{<Kiểu 2>}
\setTheoBox{theorem2}{2}{\bf Kiểu số 2}
\newtheorem{theorem3}{<Kiểu 3>}
\setTheoBox{theorem3}{3}{\bf Kiểu số 3}
\newtheorem{theorem4}{<Kiểu 4>}
\setTheoBox{theorem4}{4}{\bf Kiểu số 4}
\newtheorem{theorem5}{<Kiểu 5>}
\setTheoBox{theorem5}{5}{\bf Kiểu số 5}
\newtheorem{theorem6}{<Kiểu 6>}
\setTheoBox{theorem6}{6}{\bf Kiểu số 6}
\newtheorem{theorem7}{<Kiểu 7>}
\setTheoBox{theorem7}{7}{\bf Kiểu số 7}
\newtheorem{theorem8}{<Kiểu 8>}
\setTheoBox{theorem8}{8}{\bf Kiểu số 8}
\newtheorem{theorem9}{<Kiểu 9>}
\setTheoBox{theorem9}{9}{\bf Kiểu số 9}
\newtheorem{theorem10}{<Kiểu 10>}
\setTheoBox{theorem10}{10}{\bf Kiểu số 10}
\newtheorem{theorem11}{<Kiểu 11>}
\setTheoBox{theorem11}{11}{\bf Kiểu số 11}
\newtheorem{theorem12}{<Kiểu 12>}
\setTheoBox{theorem12}{12}{\bf Kiểu số 12}
\newtheorem{theorem13}{<Kiểu 13>}
\setTheoBox{theorem13}{13}{\bf Kiểu số 13}
\newtheorem{theorem14}{<Kiểu 14>}
\setTheoBox{theorem14}{14}{\bf Kiểu số 14}
\newtheorem{theorem15}{<Kiểu 15>}
\setTheoBox{theorem15}{15}{\bf Kiểu số 15}
\newtheorem{theorem16}{<Kiểu 16>}
\setTheoBox{theorem16}{16}{\bf Kiểu số 16}
\newtheorem{theorem17}{<Kiểu 17>}
\setTheoBox{theorem17}{17}{\bf Kiểu số 17}
\newtheorem{theorem18}{<Kiểu 18>}
\setTheoBox{theorem18}{18}{\bf Kiểu số 18}
\newtheorem{theorem19}{<Kiểu 19>}
\setTheoBox{theorem19}{19}{\bf Kiểu số 19}
\newtheorem{theorem20}{<Kiểu 20>}
\setTheoBox{theorem20}{20}{\bf Kiểu số 20}
\newtheorem{theorem21}{<Kiểu 21>}
\setTheoBox{theorem21}{21}{\bf Kiểu số 21}
\newtheorem{theorem22}{<Kiểu 22>}
\setTheoBox{theorem22}{22}{\bf Kiểu số 22}
\newtheorem{theorem23}{<Kiểu 23>}
\setTheoBox{theorem23}{23}{\bf Kiểu số 23}
\newtheorem{theorem24}{<Kiểu 24>}
\setTheoBox{theorem24}{24}{\bf Kiểu số 24}
\newtheorem{theorem25}{<Kiểu 25>}
\setTheoBox{theorem25}{25}{\bf Kiểu số 25}
\newtheorem{theorem26}{<Kiểu 26>}
\setTheoBox{theorem26}{26}{\bf Kiểu số 26}
\newtheorem{theorem27}{<Kiểu 27>}
\setTheoBox{theorem27}{27}{\bf Kiểu số 27}
\newtheorem{theorem28}{<Kiểu 28>}
\setTheoBox{theorem28}{28}{\bf Kiểu số 28}
\newtheorem{theorem29}{<Kiểu 29>}
\setTheoBox{theorem29}{29}{\bf Kiểu số 29}
%-------------------------------------------
\newsavebox{\myverbcontent}
%---------------- Bắt đầu ------------------
\begin{document}
	\pagestyle{plain}
	\fontsize{12pt}{16pt}\selectfont
	\begin{center}
		\Large\bf\color{red!70!black}\fontfamily{lmss}\selectfont Hướng dẫn sử dụng gói lệnh set\_box.sty 1.2025.08.24\par
	\end{center}
	\tableofcontents
	\newpage
	\section{Giới thiệu sơ lược về gói lệnh}
	\subsection{Nguyên nhân ra đời}
	\begin{itemize}
		\item Đơn giản hoá bước tạo các khung nội dung mới theo mẫu có sẵn.
		\item Các khung được tạo tương thích tốt với gói lệnh \texttt{ex\_test.sty} đã rất phổ biến hiện nay.
		\item Giải quyết được các vấn đề về lồng môi trường vào nhau, cũng như ẩn hiện môi trường được đóng khung.
	\end{itemize}
	\subsection{Một số lưu ý}
	\begin{itemize}
		\item Gói lệnh nên đi kèm và khai báo phía sau hai gói \texttt{ex\_test.sty} và \texttt{ntheorem.sty}. 
		\item Chỉ nên cài đặt khung cho theorem mới hoặc đã áp dụng khung trước đó. Hạn chế áp dụng thêm các tác động khác ngoài gói này, có thể gây lỗi.
	\end{itemize}
	\subsection{Đôi lời muốn nói}
	\begin{itemize}
		\item Hiện tại gói lệnh vẫn chưa phát triển câu lệnh để thầy cô tự tạo khung riêng như lệnh \verb|\createbox| của gói lệnh \texttt{ex\_test}.
	\end{itemize}
	\newpage
	\section{Hướng dẫn sử dụng}
	\subsection{Cài đặt khung cho \texttt{theorem} (định nghĩa bởi gói \texttt{ntheorem})}
	\subsubsection{Giới thiệu lệnh}
	\begin{itemize}
		\item \textbf{Bước 1.} Tạo \texttt{theorem} mới bằng gói lệnh \texttt{ntheorem}.\par
		\begin{lrbox}{\myverbcontent}
			\verb|\setTheoBox{<tên theorem>}{<kiểu>}{<Tiêu đề>}|
		\end{lrbox}
		\item \textbf{Bước 2.} Dùng lệnh \fbox{\usebox{\myverbcontent}} để cài đặt khung cho \texttt{theorem} vừa tạo theo kiểu mong muốn.\par
	\end{itemize}
	\subsubsection{Lưu ý}
	\begin{itemize}
		\item Mọi \texttt{theorem} mà nội dung có chứa \verb|\loigiai| đều phải áp dụng lệnh này. Nếu không muốn tạo khung thì dùng kiểu số 0. 
		\item Lời giải mặc định được đưa ra ngoài khung.
	\end{itemize}
	\subsubsection{Minh hoạ}
	\begin{verbatim}
		\documentclass[12pt,a4paper,twoside]{article}
		\usepackage[utf8]{vietnam}
		\usepackage[left=2cm,right=2cm,top=2cm,bottom=2cm]{geometry}
		\usepackage{ntheorem}
		\usepackage[loigiai]{ex_test}
		\usepackage{set_box}
		%---------- Định nghĩa -----------
		\newtheorem{dn}{Định nghĩa}
		\setTheoBox{dn}{2}{\bf Định nghĩa}
		%------------- Ví dụ -------------
		\newtheorem{vd}{Ví dụ}
		\setTheoBox{vd}{4}{\bf Ví dụ}
		%-------- Nội dung chính ---------
		\begin{document}
			\begin{dn}[Tên định nghĩa]
				Nội dung Định nghĩa...
			\end{dn}
			\begin{vd}
				Đề bài...
				\loigiai{
					Lời giải...
				}
			\end{vd}
		\end{document}
	\end{verbatim}
	\subsubsection{Kết quả thu được}
	\begin{dn}[Tên định nghĩa]
		Nội dung Định nghĩa...
	\end{dn}
	\begin{vd}
		Đề bài...
		\loigiai{
			Lời giải...
		}
	\end{vd}
	\subsection{Thay đổi màu khung và màu nền}
	\subsubsection{Giới thiệu lệnh}
	Để đổi màu mặc định, trước \verb|\setTheoBox| và \verb|\setEnvBox|, ta dùng các lệnh sau:
	\begin{itemize}
		\begin{lrbox}{\myverbcontent}
			\verb|\setboxColframeSetDefault{<màu>}|
		\end{lrbox}
		\item Lệnh \fbox{\usebox{\myverbcontent}}: đổi màu khung mặc định.
		\begin{lrbox}{\myverbcontent}
			\verb|\setboxColbackSetDefault{<màu>}|
		\end{lrbox}
		\item Lệnh \fbox{\usebox{\myverbcontent}}: đổi màu nền mặc định.
		\begin{lrbox}{\myverbcontent}
			\verb|\setboxColbacktitleSetDefault{<màu>}|
		\end{lrbox}
		\item Lệnh \fbox{\usebox{\myverbcontent}}: đổi màu nền tiêu đề mặc định.
	\end{itemize}\par
	Để đổi màu riêng, sau \verb|\setTheoBox| và \verb|\setEnvBox|, ta dùng các lệnh sau:
	\begin{itemize}
		\begin{lrbox}{\myverbcontent}
			\verb|\setboxColframe{<tên môi trường>}{<màu>}|
		\end{lrbox}
		\item Lệnh \fbox{\usebox{\myverbcontent}}: đổi màu khung của môi trường.
		\begin{lrbox}{\myverbcontent}
			\verb|\setboxColback{<tên môi trường>}{<màu>}|
		\end{lrbox}
		\item Lệnh \fbox{\usebox{\myverbcontent}}: đổi màu nền của môi trường.
		\begin{lrbox}{\myverbcontent}
			\verb|\setboxColbacktitle{<tên môi trường>}{<màu>}|
		\end{lrbox}
		\item Lệnh \fbox{\usebox{\myverbcontent}}: đổi màu nền tiêu đề của môi trường.
	\end{itemize}\par
	\subsubsection{Minh hoạ}
	\begin{verbatim}
		\documentclass[12pt,a4paper,twoside]{article}
		\usepackage[utf8]{vietnam}
		\usepackage[left=2cm,right=2cm,top=2cm,bottom=2cm]{geometry}
		\usepackage{ntheorem}
		\usepackage[loigiai]{ex_test}
		\usepackage{set_box}
		%--------- Tạo môi trường Khởi động -----------
		\newtheorem{kd}{Khởi động}
		\setTheoBox{kd}{14}{\bf Khởi động}
		\setboxColframe{kd}{red}
		\setboxColback{kd}{yellow!20}
		\setboxColbacktitle{kd}{orange}
		%--------- Tạo môi trường Tính chất -----------
		\newtheorem{tc}{Tính chất}
		\setTheoBox{tc}{1}{\bf Tính chất}
		\setboxColframe{tc}{brown}
		\setboxColback{tc}{blue!10}
		%------- Nội dung chính --------
		\begin{document}
			\begin{kd}
				Nội dung Khởi động...
			\end{kd}
			\begin{tc}[Tên tính chất]
				Nội dung Tính chất...
			\end{tc}
		\end{document}
	\end{verbatim}
	\subsubsection{Kết quả thu được}
	\begin{kd}
		Nội dung Khởi động...
	\end{kd}
	\begin{tc}[Tên tính chất]
		Nội dung Tính chất...
	\end{tc}
	\newpage
	\subsection{Ẩn khung tự động khi sử dụng môi trường bên trong \texttt{tcolorbox}}
	\subsubsection{Giới thiệu lệnh}
	\begin{itemize}
		\begin{lrbox}{\myverbcontent}
			\verb|\showboxInTcb{<tên môi trường>}|
		\end{lrbox}
		\item Lệnh \fbox{\usebox{\myverbcontent}} (mặc định): hiện ``khung'' khi sử dụng môi trường trong \verb|\begin{tcolorbox}...\end{tcolorbox}|.
		\begin{lrbox}{\myverbcontent}
			\verb|\hideboxInTcb{<tên môi trường>}|
		\end{lrbox}
		\item Lệnh \fbox{\usebox{\myverbcontent}}: ẩn ``khung'' khi sử dụng môi trường trong trong \verb|\begin{tcolorbox}...\end{tcolorbox}|.
	\end{itemize}\par
	\subsubsection{Minh hoạ}
	\begin{verbatim}
		\hideboxInTcb{dl} % Ẩn khung
		\hideboxInTcb{btm} % Ẩn khung
		\begin{tcolorbox}[frame empty,colback=red!5]
			\begin{dl}[Tên định lý]
				Nội dung Định lý...
			\end{dl}
			\begin{btm}
				Đề bài...
				\loigiai{
					Lời giải...
				}
			\end{btm}
		\end{tcolorbox}
		\showboxInTcb{dl} % Hiện khung
		\showboxInTcb{btm} % Hiện khung
		\begin{tcolorbox}[frame empty,colback=red!5]
			\begin{dl}[Tên định lý]
				Nội dung Định lý...
			\end{dl}
			\begin{btm}
				Đề bài...
				\loigiai{
					Lời giải...
				}
			\end{btm}
		\end{tcolorbox}
	\end{verbatim}
	\newpage
	\subsubsection{Kết quả thu được}
	\setcounter{dl}{0}
	\setcounter{btm}{0}
	\hideboxInTcb{dl}
	\hideboxInTcb{btm}
	\begin{tcolorbox}[frame empty,colback=red!5]
		\begin{dl}[Tên định lý]
			Nội dung Định lý...
		\end{dl}
		\begin{btm}
			Đề bài...
			\loigiai{
				Lời giải...
			}
		\end{btm}
	\end{tcolorbox}
	\showboxInTcb{dl}
	\showboxInTcb{btm}
	\begin{tcolorbox}[frame empty,colback=red!5]
		\begin{dl}[Tên định lý]
			Nội dung Định lý...
		\end{dl}
		\begin{btm}
			Đề bài...
			\loigiai{
				Lời giải...
			}
		\end{btm}
	\end{tcolorbox}
	\subsubsection{Lưu ý}
	Để tạo một \texttt{tcolorbox} nền mà các lệnh \verb|\showboxInTcb| và \verb|\hideboxInTcb| \textit{có thể} tác động lên các môi trường bên trong, ta có thể thử sử dụng cấu trúc khai báo sau để chèn \verb|\begin{tcolorbox}| và \verb|\end{tcolorbox}|:
	\begin{verbatim}
		% Khai báo bắt đầu tcolorbox cho môi trường nen (Nền)
		\def\beginNen{
			\begin{tcolorbox}[<option>]
			}
		% Khai báo kết thúc tcolorbox cho môi trường nen (Nền)
		\def\endNen{ % Không nên dùng \endbox
			\end{tcolorbox}
		}
		% Tạo môi trường nen (Nền)
		\newenvironment{nen}{\beginNen}{\endNen}
	\end{verbatim}
	Để tạo một \texttt{tcolorbox} nền mà các lệnh \verb|\showboxInTcb| và \verb|\hideboxInTcb| \textit{không thể} tác động lên các môi trường bên trong, ta có thể thử sử dụng cấu trúc khai báo sau để chèn \verb|\begin{tcolorbox}| và \verb|\end{tcolorbox}|:
	\begin{verbatim}
		% Tạo tcolorbox mới nen (Nền)
		\newtcolorbox{nen}{<option>}
	\end{verbatim}
	\subsection{Ẩn môi trường}
	\subsubsection{Giới thiệu lệnh}
	\begin{itemize}
		\begin{lrbox}{\myverbcontent}
			\verb|\setboxHide{<tên môi trường>}|
		\end{lrbox}
		\item Lệnh \fbox{\usebox{\myverbcontent}}: ẩn môi trường hoàn toàn kể từ khi áp dụng lệnh này.
		\begin{lrbox}{\myverbcontent}
			\verb|\setboxHide[<danh sách>]{<tên môi trường>}|
		\end{lrbox}
		\item Lệnh \fbox{\usebox{\myverbcontent}}: ẩn môi trường theo danh sách kể tử khi áp dụng lệnh này.
	\end{itemize}
	\subsubsection{Lưu ý}
	\begin{itemize}
		\item Chỉ số sau khi ẩn vẫn đánh giống như chỉ số gốc khi không ẩn.
		\item Hai lệnh này là lệnh mới của gói \texttt{set\_box}, không phải là lệnh \verb|\hideenviron| và \verb|\print| của \texttt{ex\_test}, chỉ áp dụng với các môi trường đã được xử lý bởi \verb|\setTheoBox| hoặc \verb|\setEnvBox|.
		\item Khi cần ẩn các câu hỏi, ta có thể khai báo danh sách tương tự khi dùng \verb|\foreach|.Chẳng hạn, danh sách gồm các số từ 3 đến 5 và 10 đến 15 thì ta dùng
		\begin{center}
			\verb|\setboxHide[3,...,5,10,...,15]{<tên môi trường>}|.
		\end{center}
	\end{itemize}\par
	\subsubsection{Minh hoạ}
	\begin{verbatim}
		\setcounter{dn}{0}
		\setboxHide{dn} % Lệnh ẩn môi trường Định nghĩa
		\begin{dn}
			Nội dung của Định nghĩa đầu tiên...
		\end{dn}
		\begin{dn}
			Nội dung của Định nghĩa thứ hai...
		\end{dn}
		\setcounter{dn}{0}
		\setboxHide[2,4]{dn} % Lệnh ẩn môi trường Định nghĩa
		\begin{dn}
			Nội dung của Định nghĩa 1...
		\end{dn}
		\begin{dn}
			Nội dung của Định nghĩa 2...
		\end{dn}
		\begin{dn}
			Nội dung của Định nghĩa 3...
		\end{dn}
		\begin{dn}
			Nội dung của Định nghĩa 4...
		\end{dn}
	\end{verbatim}
	\subsubsection{Kết quả thu được}
	\setcounter{dn}{0}
	\setboxHide{dn} % Lệnh ẩn Định nghĩa
	\begin{dn}
		Nội dung của Định nghĩa đầu tiên...
	\end{dn}
	\setcounter{dn}{0}
	\setboxHide[2,4]{dn} % Lệnh ẩn Định nghĩa
	\begin{dn}
		Nội dung của Định nghĩa 1...
	\end{dn}
	\begin{dn}
		Nội dung của Định nghĩa 2...
	\end{dn}
	\begin{dn}
		Nội dung của Định nghĩa 3...
	\end{dn}
	\begin{dn}
		Nội dung của Định nghĩa 4...
	\end{dn}
	\newpage
	\section{Các kiểu khung có sẵn ở phiên bản hiện tại}
	\begin{theorem0}
		Nội dung của kiểu 0...
	\end{theorem0}
	\begin{theorem1}
		Nội dung của kiểu 1...
	\end{theorem1}
	\begin{theorem2}
		Nội dung của kiểu 2...
	\end{theorem2}
	\begin{theorem3}
		Nội dung của kiểu 3...
	\end{theorem3}
	\begin{theorem4}
		Nội dung của kiểu 4...
	\end{theorem4}
	\begin{theorem5}
		Nội dung của kiểu 5...
	\end{theorem5}
	\begin{theorem6}
		Nội dung của kiểu 6...
	\end{theorem6}
	\begin{theorem7}
		Nội dung của kiểu 7...
	\end{theorem7}
	\begin{theorem8}
		Nội dung của kiểu 8...
	\end{theorem8}
	\begin{theorem9}
		Nội dung của kiểu 9...
	\end{theorem9}
	\begin{theorem10}
		Nội dung của kiểu 10...
	\end{theorem10}
	\begin{theorem11}
		Nội dung của kiểu 11...
	\end{theorem11}
	\begin{theorem12}
		Nội dung của kiểu 12...
	\end{theorem12}
	\begin{theorem13}
		Nội dung của kiểu 13...
	\end{theorem13}
	\begin{theorem14}
		Nội dung của kiểu 14...
	\end{theorem14}
	\begin{theorem15}
		Nội dung của kiểu 15...
	\end{theorem15}
	\begin{theorem16}
		Nội dung của kiểu 16...
	\end{theorem16}
	\begin{theorem17}
		Nội dung của kiểu 17...
	\end{theorem17}
	\begin{theorem18}
		Nội dung của kiểu 18...
	\end{theorem18}
	\begin{theorem19}
		Nội dung của kiểu 19...
	\end{theorem19}
	\begin{theorem20}
		Nội dung của kiểu 20...
	\end{theorem20}
	\begin{theorem21}
		Nội dung của kiểu 21...
	\end{theorem21}
	\begin{theorem22}
		Nội dung của kiểu 22...
	\end{theorem22}
	\begin{theorem23}
		Nội dung của kiểu 23...
	\end{theorem23}
	\begin{theorem24}
		Nội dung của kiểu 24...
	\end{theorem24}
	\begin{theorem25}
		Nội dung của kiểu 25...
	\end{theorem25}
	\begin{theorem26}
		Nội dung của kiểu 26...
	\end{theorem26}
	\begin{theorem27}
		Nội dung của kiểu 27...
	\end{theorem27}
	\begin{theorem28}
		Nội dung của kiểu 28...
	\end{theorem28}
	\begin{theorem29}
		Nội dung của kiểu 29...
	\end{theorem29}
	\newpage
	\section{Ví dụ hoàn chỉnh}
	\subsection{Minh hoạ}
	\begin{verbatim}
		\documentclass[12pt,a4paper,twoside]{article}
		\usepackage[utf8]{vietnam}
		\usepackage[left=2cm,right=2cm,top=2cm,bottom=2cm]{geometry}
		\usepackage{xcolor}
		\usepackage{ntheorem}
		\usepackage[loigiai]{ex_test}
		\usepackage{set_box}
		\usepackage{indentfirst}
		\usepackage{lipsum}
		\usepackage{needspace}
		%---- Cài đặt màu mặc định gói set_box ----
		\setboxColframeSetDefault{brown}
		\setboxColbackSetDefault{black!5}
		\setboxColbacktitleSetDefault{brown!30}
		%----------- Định dạng theorem ------------
		\theoremstyle{immini}
		%--------------- Định nghĩa ---------------
		\newtheorem{dn}{Định nghĩa}
		\setTheoBox{dn}{23}{\color{brown!70!black}\bf\fontfamily{lmss}
		\selectfont Định nghĩa}
		%--------------- Tính chất ---------------
		\newtheorem{tc}{Tính chất}
		\setTheoBox{tc}{26}{\color{brown!70!black}\bf\fontfamily{lmss}
		\selectfont Tính chất}
		%----------------- Ví dụ -----------------
		\newtheorem{vd}{Ví dụ}
		\setTheoBox{vd}{22}{\color{brown!70!black}\bf\fontfamily{lmss}
		\selectfont Ví dụ}
		%----------------- Dạng- -----------------
		\newtheorem{dang}{Dạng}
		\setTheoBox{dang}{11}{\color{brown!70!black}\bf\fontfamily{lmss}
		\selectfont Dạng}
		%---------------- Ghi nhớ ----------------
		\newtheorem{ghinho}{Ghi nhớ}
		\setTheoBox{ghinho}{21}{\color{brown!70!black}\bf\fontfamily{lmss}
		\selectfont Ghi nhớ}
		%------------------- ex ------------------
		\renewtheorem{ex}{Câu}
		\setTheoBox{ex}{14}{\color{brown!70!black}\bf\fontfamily{lmss}
		\selectfont Câu}
		% Gói lệnh sẽ áp dụng tương tự đối với chc
		%----------------- Lời giải --------------
		\def\loigiaiEX{\needspace{1cm}\parbox[c]{\linewidth-1mm}{\centering
		\color{brown!70!black}\bf\fontfamily{lmss}\selectfont Lời giải.}}
		\theorembodyfont{\itshape}
		%----------------- Chú ý -----------------
		\newtheorem{chuy}{Chú ý}
		\setTheoBox{chuy}{24}{\color{brown!70!black}\bf\fontfamily{lmss}
		\selectfont Chú ý}
		%--------------- Nhận xét ----------------
		\newtheorem{nx}{Nhận xét}
		\setTheoBox{nx}{5}{\color{brown!70!black}\bf\fontfamily{lmss}
		\selectfont Nhận xét}
		\begin{document}
			\begin{dn}[Tên định nghĩa]
				\lipsum[1]
			\end{dn}
			\begin{dn}[Tên định nghĩa]
				\lipsum[1]
			\end{dn}
			\begin{tc}[Tên tính chất]
				\lipsum[1]
			\end{tc}
			\begin{vd}
				\lipsum[1]
				\loigiai{
					\dotlineEX{3}
				}
			\end{vd}
			\begin{vd}
				\lipsum[1]
				\loigiai{
					\dotlineEX{3}
				}
			\end{vd}
			\begin{chuy}
				\begin{itemize}
					\item Điều thứ nhất;
					\item Điều thứ hai.
				\end{itemize}
			\end{chuy}
			\begin{chuy}
				\begin{itemize}
					\item Điều thứ nhất;
					\item Điều thứ hai.
				\end{itemize}
			\end{chuy}
			\begin{nx}
				\lipsum[1]
			\end{nx}
			\begin{dang}[Tên dạng]
				\lipsum[1]
			\end{dang}
			
			\begin{ex}
				\immini[thm]{
					Nội dung câu hỏi con thứ nhất.
				}{
					\begin{tikzpicture}
						\draw (0,0) -- (3,3);
					\end{tikzpicture}
				}
				\begin{chc}
					Nội dung câu hỏi con thứ nhất.
					\loigiai{
						<Lời giải câu hỏi con thứ nhất>
					}
				\end{chc}
				\begin{chc}
					Nội dung câu hỏi con thứ hai.
					\loigiai{
						<Lời giải câu hỏi con thứ hai>
					}
				\end{chc}
			\end{ex}
			\begin{ex}
				\sochc{2}
				\immini[thm]{
					Nội dung câu hỏi con thứ nhất.
				}{
					\begin{tikzpicture}
						\draw (0,0) -- (3,3);
					\end{tikzpicture}
				}
				\begin{chc}
					Nội dung câu hỏi con thứ nhất.
					\loigiai{
						<Lời giải câu hỏi con thứ nhất>
					}
				\end{chc}
				\begin{chc}
					Nội dung câu hỏi con thứ hai.
					\loigiai{
						<Lời giải câu hỏi con thứ hai>
					}
				\end{chc}
			\end{ex}
			\begin{ghinho}
				\lipsum[1]
			\end{ghinho}
			\begin{ex}
				<Câu dẫn cho câu hỏi nhiều lựa chọn>.
				\choice
				{\True Phương án A}
				{Phương án B}
				{Phương án C}
				{Phương án D}
				\loigiai{
					...
				}
			\end{ex}
			\begin{ex}
				<Câu dẫn cho câu hỏi nhiều lựa chọn>.
				\choiceTF[1t]
				{\True Phương án A}
				{Phương án B}
				{Phương án C}
				{\True Phương án D}
				\loigiai{
					...
				}
			\end{ex}
		\end{document}
	\end{verbatim}
	\subsection{Kết quả thu được}
	Xem ở trang tiếp theo.
	\newpage
	\includepdf[pages=-]{example.pdf}
\end{document}